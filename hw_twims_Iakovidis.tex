\documentclass[a4paper]{article} % this is used for comments
\usepackage[utf8]{inputenc}
%%% Дополнительная работа с математикой
\usepackage{amsmath,amsfonts,amssymb,amsthm,mathtools} % AMS
\usepackage{icomma} % "Умная" запятая: $0,2$ --- число, $0, 2$ --- перечисление
\usepackage[english,russian]{babel}

%% Номера формул
\mathtoolsset{showonlyrefs=true} % Показывать номера только у тех формул, на которые есть \eqref{} в тексте.

%% Шрифты
\usepackage{euscript}	 % Шрифт Евклид
\usepackage{mathrsfs} % Красивый матшрифт

\usepackage{physics}

%% Свои команды
\DeclareMathOperator{\sgn}{\mathop{sgn}}

%% Перенос знаков в формулах (по Львовскому)
\newcommand*{\hm}[1]{#1\nobreak\discretionary{}
{\hbox{$\mathsurround=0pt #1$}}{}}

\DeclareMathOperator{\Lin}{\mathrm{Lin}}
\DeclareMathOperator{\Linp}{\Lin^{\perp}}
\DeclareMathOperator*\plim{plim}
\DeclareMathOperator{\card}{card}
\DeclareMathOperator{\sign}{sign}

\DeclareMathOperator*{\argmin}{arg\,min}
\DeclareMathOperator*{\argmax}{arg\,max}
\DeclareMathOperator*{\amn}{arg\,min}
\DeclareMathOperator*{\amx}{arg\,max}
\DeclareMathOperator{\cov}{Cov}
\DeclareMathOperator{\Var}{Var}
\DeclareMathOperator{\Cov}{Cov}
\DeclareMathOperator{\Corr}{Corr}
\DeclareMathOperator{\pCorr}{pCorr}
\DeclareMathOperator{\E}{\mathbb{E}}
\let\P\relax
\DeclareMathOperator{\P}{\mathbb{P}}

\newcommand{\cN}{\mathcal{N}}
\newcommand{\cU}{\mathcal{U}}
\newcommand{\cBinom}{\mathcal{Binom}}
\newcommand{\cBin}{\cBinom}
\newcommand{\cPois}{\mathcal{Pois}}
\newcommand{\cBeta}{\mathcal{Beta}}
\newcommand{\cGamma}{\mathcal{Gamma}}

\newcommand \R{\mathbb{R}}
\newcommand \N{\mathbb{N}}
\newcommand \Z{\mathbb{Z}}

\newcommand{\dx}[1]{\,\mathrm{d}#1} % для интеграла: маленький отступ и прямая d
\newcommand{\ind}[1]{\mathbbm{1}_{\{#1\}}} % Индикатор события
%\renewcommand{\to}{\rightarrow}
\newcommand{\eqdef}{\mathrel{\stackrel{\rm def}=}}
\newcommand{\iid}{\mathrel{\stackrel{\rm i.\,i.\,d.}\sim}}
\newcommand{\const}{\mathrm{const}}

% вместо горизонтальной делаем косую черточку в нестрогих неравенствах
\renewcommand{\le}{\leqslant}
\renewcommand{\ge}{\geqslant}
\renewcommand{\leq}{\leqslant}
\renewcommand{\geq}{\geqslant}





\title{Промежуточный экзамен 2016 - 2017}
\author{БЭК182, Яковидис Одиссей}
\date{Июнь 2020}






\begin{document}

\maketitle

\text{Be good, drink milk and think of Russia}
\vspace{\baselineskip}

\textbf{Промежуточный экзамен 2016-2017}

\textbf{Ответы:}
BCAAE DBDBD AACA? ?EECB ?ABAB DAEDA CD


\begin{enumerate}

    \item
    Когда граф Сен-Жермен извлекает из колоды первую карту, общее количество карт в колоде уменьшается. Вероятность события $B$ соответственно меняется и становится условной вероятностью $\P(B\mid A)$. 
    Аналогично для события С

    Таким образом, события $A$, $B$ и $C$ не могут являться независимыми.
    
    Тот же результат можно получить просто посчитав вероятности соответствующих событий.
    
    Ответ: B
    
    \item
    
    Заметим, что среди всех данных функций только функция $C$ обладает свойствами неотрицательности и непрерывности с области определения притом, что

    \[\int_{-\infty}^{+\infty} f(x) dx = \int_{1}^{+\infty} dx/x^2 = 1\] 
    
   Тогда такая функция может являться функцией плотности с.в.
    
    Ответ: C
    
    \item
    
    По формуле для нахождения ковариации через м.о. :
    \[\Cov(X,Y) = \E(XY) - \E(X)\E(Y)\]
    \[\E(XY) = 2 + 3\cdot2 = 8\]
    
    Ответ: A
    
    \item 
    Корреляцию двух случайных величин можно вычислить как:
    \[\Corr(X,Y) = \frac{\Cov(X,Y)}{\sqrt{\Var(X)} \cdot \sqrt{\Var(Y)}} = \frac{2}{\sqrt{12} \cdot \sqrt{1}} = \frac{1}{\sqrt{3}}\]
    
    Ответ: A
    
    \item
    По свойствам дисперсии:
    \[\Var(aX + bY + c) = a^2 \Var(X) + b^2 \Var(Y) + 2ab\Cov(X,Y)\]
    \[\Var(2X - Y + 4) = 4 \cdot 12 + 1 \cdot 1 - 4 \cdot 2 = 48 + 1 - 8 = 41\]
    
    Ответ: E
    
    \item
    Заметим, что для такой ковариационной матрицы двух случайных величин: $\Var(X) = 1$, $\Var(Y) = 1$, $\Cov(X,Y) = 0$
    
    Из нулевой ковариации следует независимость $X$ и $Y$
    
    Ответ: D

    \item
    По свойствам корреляции, ковариации и дисперсий:
    
    \[Corr(X, Y) = \frac{\Cov(X, Y)}{\sqrt{\Var(X)}\sqrt{\Var(Y)}}\]
    
    Тогда подставляя исходные данные получаем: 
    
    \[\Var(X) = \Var(Y) = 2 \Cov(X, Y)\]
    
    \[\Corr(X+Y, 2Y-7) = \frac{\Cov(X+Y, 2Y-7)}{\sqrt{\Var(X+Y)}\sqrt{\Var(2Y-7)}}\] 
    
    \[\Cov(X+Y, 2Y-7) = \Cov(X, 2Y) + 2 \Var(Y) = 6\Cov(X, Y)\]
    
    \[\Var(X + Y) = \Var(X) + \Var(Y) + 2\Cov(X;Y) = 6\Cov(X; Y)\]
    
    \[\Var(2Y - 7) = 4\Var(Y) = 8\Cov(X; Y)\]
    
    \[\Corr(X+Y, 2Y-7) = \frac{6\Cov(X; Y)}{\sqrt{48\cdot\Cov(X; Y)}} = \frac{6}{4\sqrt{3}} = \frac{2}{2\sqrt{3}} = \frac{\sqrt{3}}{2}\]
    
    Ответ: B

    \item
    
    По свойствам равномерного распределения на отрезке [0; 1]:
    
    \[\P(0.2 < \xi < 0.7) = \frac{0.7 - 0.2}{1 - 0} = 0.5\]
    
    Ответ: D
    
    \item
    По центральной предельной теореме, указанное распределение асимптотически сходится к стандартному нормальному распределению $\cN(0,1)$ 
    
    Из предложенных интегралов только $B$ имеет верные пределы интегрирования и соответствующую функцию плотности
    
    Ответ: B
    
    \item
    
    Вычислим м.о. и дисперсию искомой суммы посетителей сайта:
    
    \[\E(S_n) = \E(X_i) \cdot n = 400 \cdot 100 = 40000\]
    
    \[\Var(S_n) = \Var\left(\sum X_i\right) = n \cdot \Var(X_i) = 40000\]
    
    Стандартизируем случайную величину $S_n$ и получаем ответ:
    
    \begin{align*}
    \P(S_n > 40400) = \P\left(\frac{S_n - \E(S_n)}{\sqrt{\Var(S_n)}} > \frac{40400 - 40000}{\sqrt{40000}}\right) = \P(Z > 2)= \\
    = 1 -\P(Z < 2) = 1 - 0.9772 = 0.0227
    \end{align*}
    
    Ответ: D
    
    \item
    $X$ - неотрицательная случайная величина. 
    
    Тогда согласно неравенству Маркова:

    \[\P({X} \leq a) \leq \frac{\E(X)}{a}\]
    
    \[\P({X} \leq 50000) \leq \frac{10000}{50000} = 0.2\]
    
    Ответ: A
    
    \item
    
    Заметим, что событие $A$ и событие $B$ для трех бросков монеты могут произойти одновременно. Однако это нельзя сказать про события $A$ и $C$, так как если из трех бросков монеты все три раза выпал орел, вероятность выпадения решки равна 0.
    
    Тогда события $A$ и $B$ совметсны, а события $A$ и $C$ несовместны.
    
    Ответ: A
    
    \item

    Согласно неравенству Чебышёва:
    
    \[\P(\abs{X - \E(X)} > a) \leq \frac{\Var(X)}{a^2}\]
    
    Тогда:
    \begin{align*}
    \P(\abs{X- 50000} \leq 20000) = 1 - \P(\abs{X- 50000} > 20000) = \\
    =1 - \frac{10^8}{4 \cdot 10^8} = 1 - \frac{1}{4} = \frac{3}{4}
    \end{align*}
    
    Ответ: C
    
    \item
    Случайная величина $\xi$ имеет биномиальное распределение.
    Тогда:
    \[\E(\xi_i) = 0.6\]
    
    По закону больших чисел:
    \[\plim_{n\to\infty} \frac{(\xi_1)^{2016} + \ldots + (\xi_n)^{2016}}{n} = \E((\xi_1)^{2016}) = 0.6\]
    
    Ответ: A
    
    \item
    По свойству биномиального распределения:
    \[\P(X=2) = C_5^2 \cdot\left(\frac{1}{6}\right)^2 \cdot \left(\frac{5}{6}\right)^3 = \frac{5^4}{2^4\cdot3^5}\]
    
    Ответ: $\frac{5^4}{2^4\cdot3^5}$
    
    \item
    Для биномиального распределения:
   
    Математическое ожидание шестерок в 5 бросках
    
    \[\E(5X) = 5\E(X) = \frac{5}{6}\]
    
    Дисперсия
    \[\Var = np(1-p) = 5 \cdot \frac{1}{6} \cdot \frac{5}{6} = \frac{25}{36}\]
    
    Ответ: 5/6 и 25/36
    
    \item
    Для биномиального распределения (m - наиболее вероятное число):
    \[np-q \leq m \leq np+p\]
    \[5 \cdot \frac{1}{6} - \frac{5}{6} \leq m \leq 5 \cdot \frac{1}{6} + \frac{1}{6}\]
    \[0 \leq m \leq 1\]
    
    Ответ: E
    
    \item
    Для одного броска:
    \[\E(X) = \frac{1}{6}\cdot(1 + 2 + 3 + 4 + 5 + 6) = \frac{21}{6}\]
    Для 5 бросков:
    \[\E(5X) = 5\E(X) = 5 \cdot \frac{21}{6} = 17.5\]
    
    Ответ: E
    
    \item
    Для решения этой задачи необходимо вспомнить свойства нормального двумерного распределения.
    Из условия задачи:
    \[\E(\xi) = \E(\eta) = 0\]
    \[\Var(\xi) = \Var(\eta) = 1\]
    \[\Cov(\xi, \eta) = \Cov(\eta, \xi) = 0.5\]
    
    Тогда:
    \[\Corr(\xi, \eta) = \frac{\Cov(\xi, \eta)}{\sqrt{\Var(\xi)}\sqrt{\Var(\eta)}} = 0.5\]
    
    Подставив наши параметры в функцию плотности нормального двумерного распределения получаем:

    \[a = \sqrt{(1-0.5^2)} = \sqrt{0.75} = \frac{\sqrt{3}}{2}; b=1\] 
    
    Ответ: C
    
    \item
    Обе компоненты вектора содержат нормальные случайные величины или являются их линейной комбинацией.
    
    Тогда $z$ - двумерный нормальный вектор

    Ответ: B

    \item
    %Let`s skip

    
    \item
    По определению условного м.о. :
    \[\E(X \mid Y=0) = 0 \cdot \P(X=0 \mid Y=0) + 2 \cdot \P(X=2 \mid Y=0) = 0 \cdot \frac{1}{2} + 2 \cdot \frac{1}{2} = 1\]
    
    Ответ: A
    
    \item
    По определению условной вероятности:
    \[\P(X=0 \mid Y<1) = \frac{\P(X=0 \cap Y<1)}{\P(Y<1)} = \frac{\frac{1}{6}}{\frac{1}{3}+\frac{1}{6}+\frac{1}{6}} = \frac{1}{4}\]
    
    Ответ: B

    \item
    Из таблицы вычисляем:

    \[\E^2(Y) = (\frac{1}{3} \cdot (-1) + \frac{1}{3} \cdot 0 + \frac{1}{3} \cdot 1)^2 = 0\]

    \[\E(Y^2) = \frac{1}{3} \cdot (-1)^2 + \frac{1}{3} \cdot 0^2 + \frac{1}{3} \cdot 1^2 = \frac{2}{3}\]

    \[\Var(Y) = \E(Y^2) - \E^2(Y) = \frac{2}{3} - 0 = \frac{2}{3}\]
    
    Ответ: A
    
    \item
    По формуле ковариации через м.о.:
    \[\Cov(XY) = \E(XY) - \E(X)\E(Y)\]
    \[\E(X) = 0\cdot\frac{1}{3} + 2\cdot\frac{2}{3} = \frac{4}{3}\]
    \[\E(Y) = \frac{1}{3} \cdot (-1) + \frac{1}{3} \cdot 0 + \frac{1}{3} \cdot 1 = 0\]
    \[\E(XY) = 0\cdot(-1)\cdot0 + 2\cdot(-1)\cdot\frac{1}{3} + 0\cdot0\cdot\frac{1}{6} + 0\cdot0\cdot\frac{1}{6} + 0\cdot1\cdot\frac{1}{6} + 2\cdot1\cdot\frac{1}{6} = -\frac{1}{3}\]
    \[\Cov(XY) = -\frac{1}{3}\]
    
    Ответ: B

    \item
    Для нахождения искомой вероятности вычислим соответствующий интеграл:
    
    \[\P(X<0.5, Y<0.5) = \int_{0}^{0.5}\int_{0}^{0.5} 9x^2y^2 \,dx\,dy = \int_{0}^{0.5} \frac{3}{8}y^2 dy = \frac{1}{64}\]
    
    Ответ: D
    
    \item
    По определению условной функции плотности:
    \[f_{x\mid y=1} = \frac{f_{xy}}{f_{y}}\]
    
    Тогда:
    \[f_{y} = \int_{0}^{1} 9x^2y^2 \,dx = 3y^2\]
    
    \[f_{x\mid y=1} = \frac{9x^2y^2}{3y^2} = 3x^2, x \in [0; 1]\]
    
    Ответ: A
    
    \item
    Искомая вероятность будет равна сумме соответствующих частных вероятностей:

    \[\P(\text{<<Отличник>>}) = 0.5\cdot\frac{1}{3} + 0.3\cdot\frac{1}{3} + 0.4\cdot\frac{1}{3} = 0.4\]
    
    Ответ: E
    
    \item
    По теореме умножения для двух событий:
    \[\P(A\cap B)=\P(A\mid B)\P(B) = 0.3 \cdot 0.5 = 0.15\]
    
    Тогда по теореме сложения:
    \[\P(A\cup B) = \P(A) + \P(B) - \P(A\cap B) = 0.2+0.5-0.15 = 0.55\]
    
    Ответ: D
    
    \item
    Для биномиального распределения:
    \[\P(\text{<<Орел>>} \geq 1) = C_{10}^1 \cdot0.2^1 \cdot 0.8^9\]
    
    Ответ: A
    
    \item
    Для ответа на вопрос посчитаем ряд вероятностей.
    
    Пусть событие $F$ -- покупателем была женщина, $S$ -- сумма чека
    
    Тогда искомая условная вероятность:
    \[\P(F \mid S>1000) = \frac{\P(F \cup S>1000)}{\P(S>1000)}\]
    
    \[\P(S>1000) = 0.5 \cdot 0.6 + 0.5 \cdot 0.3 = 0.45\]
    \[\P(F=1 \cup S>1000) = 0.5 \cdot 0.6 = 0.3\]

    \[\P(F=1 \mid S>1000) = \frac{0.3}{0.45} = \frac{2}{3}\]
    
    Ответ: C
    
    \item
    
    Из перечисленных свойств только $\P(X \in (a; b]) = F_X(b) - F_X(a)$ является свойсвтом функции распределения случайной величины.
    
    Ответ: D
    
    
\end{enumerate}


\end{document}